%Packages to include
\usepackage{graphicx}
\usepackage{multirow}
\usepackage{datetime}
\usepackage{textpos} 
\usepackage{amsmath}
\usepackage{xspace}
\usepackage{hyperref}
\usepackage{xcolor}
\usepackage{hhline}
\usepackage{colortbl}
\usepackage{tikz}
\usepackage{bm}
\usepackage{ifthen}
\usepackage{soul}
\usepackage{tabularx}
\usepackage{listings}
\usepackage{ragged2e}
\usepackage{colortbl}
\usepackage{etoolbox}
\usepackage{wasysym}
%\usepackage{transparent}
\usetikzlibrary{fit,positioning,calc,shapes,backgrounds}
\tikzset{every picture/.style={/utils/exec={\sffamily}}}%

%%%%%%%%%%%%%%%%%%%%%%%%%%%%%%%%%%%%%%%%%%%%%%%%%%%%%%%%%%%%%%%%%%%%%%%%%%%%%%%%%%%%%%%%%%%%%%%%%%%%%%%%%%%%%%%%%%%%%

\makeatletter
\newcount\beamer@sectionstartframe
\beamer@sectionstartframe=1
\apptocmd{\beamer@section}{\addtocontents{nav}{\protect\headcommand{%
            \protect\beamer@sectionframes{\the\beamer@sectionstartframe}{\the\c@framenumber}}}}{}{}
\apptocmd{\beamer@section}{\beamer@sectionstartframe=\c@framenumber\advance\beamer@sectionstartframe by1\relax}{}{}
\AtEndDocument{\immediate\write\@auxout{\string\@writefile{nav}%
        {\noexpand\headcommand{\noexpand\beamer@sectionframes{\the\beamer@sectionstartframe}{\the\c@framenumber}}}}}{}{}
\def\beamer@startframeofsection{1}
\def\beamer@endframeofsection{1}
\def\beamer@sectionframes#1#2{%
    \ifnum\c@framenumber<#1%
    \else%
    \ifnum\c@framenumber>#2%
    \else%
    \gdef\beamer@startframeofsection{#1}%
    \gdef\beamer@endframeofsection{#2}%
    \fi%
    \fi%
}
\newcommand\insertsectionstartframe{\beamer@startframeofsection}
\newcommand\insertsectionendframe{\beamer@endframeofsection}
\makeatother

\def\inserttotalsectionframenumber{%
    \pgfmathparse{(\insertsectionendframe-\insertsectionstartframe+1)}%
    \pgfmathprintnumber[fixed,precision=2]{\pgfmathresult}%
}

\def\insertsectionframenumber{%
    \pgfmathparse{(\insertframenumber-\insertsectionstartframe+1)}%
    \pgfmathprintnumber[fixed,precision=2]{\pgfmathresult}%
}

%%%%%%%%%%%%%%%%%%%%%%%%%%%%%%%%%%%%%%%%%%%%%%%%%%%%%%%%%%%%%%%%%%%%%%%%%%%%%%%%%%%%%%%%%%%%%%%%%%%%%%%%%%%%%%%%%%%%

\newcommand\Wider[2][3em]{%
\makebox[\linewidth][c]{%
  \begin{minipage}{\dimexpr\textwidth+#1\relax}%
  \raggedright#2
  \end{minipage}%
  }%
}

\newcommand\Narrower[2][3em]{%
\makebox[\linewidth][c]{%
  \begin{minipage}{\dimexpr\textwidth-#1\relax}%
  \footnotesize
  \raggedright#2
  \end{minipage}%
  }%
}

\newcommand\gridhelp{%
\draw[help lines,xstep=0.02,ystep=0.02] (0,0) grid (1,1);%
\foreach \x in {0,1,...,9} { \node[anchor=north] at (\x/10,0) {\scriptsize 0.\x}; }%
\foreach \x in {0,1,...,9} { \node[anchor=south] at (\x/10,1.0) {\scriptsize 0.\x}; }%
\foreach \y in {0,1,...,9} { \node[anchor=east]  at (0,\y/10) {\scriptsize 0.\y}; }%
\foreach \y in {0,1,...,9} { \node[anchor=west]  at (1.0,\y/10) {\scriptsize 0.\y}; }%
}

\newcommand{\arr}[1][UMNMaroon]{%
    \textcolor{#1}{%
        \raisebox{0.16\height}{%
            $\blacktriangleright$%
        }%
        \xspace\xspace\xspace%
    }%
}%

\newcommand{\HERWIG} {{\textsc{herwig}}\xspace}
\newcommand{\ttbar}{\ensuremath{t\bar{t}}\xspace}
\newcommand{\nugun}{\ensuremath{\nu\text{-gun}}\xspace}
\newcommand{\POWHEG} {{\textsc{powheg}}\xspace}
\newcommand{\MADGRAPH} {{\textsc{MadGraph}}\xspace}
\newcommand{\PYTHIA} {{\textsc{pythia}}\xspace}
\newcommand{\CMSSW} {{\textsc{CMSSW}}\xspace}

\newcommand{\MET}{\ensuremath{E_{\text{T}}^{\text{miss}}}\xspace}
\newcommand{\MPT}{\ensuremath{p_{\text{T}}^{\text{miss}}}\xspace}
\newcommand{\HT}{\ensuremath{H_{\text{T}}}\xspace}
\newcommand{\ET}{\ensuremath{E_{\text{T}}}\xspace}
\newcommand{\pt}{\ensuremath{p_{\text{T}}}\xspace}
\newcommand{\NJ}{\ensuremath{N_{\text{J}}}\xspace}
\newcommand{\MBL}{\ensuremath{M_{\text{b,l}}}\xspace}
\newcommand{\SNN}{\ensuremath{S_{NN}}\xspace}

\newcommand{\ttnopu}{\ttbar~+~0PU\xspace}
\newcommand{\ttpu}{\ttbar~+~PU\xspace}
\newcommand{\ttoot}{\ttbar~+~OOT~PU\xspace}
\newcommand{\nupu}{\ensuremath{\nu}-gun~+~PU\xspace}
\newcommand{\pinopu}{\ensuremath{\pi}-gun~+~0PU\xspace}
\newcommand{\qcdnopu}{QCD~+~0PU\xspace}
\newcommand{\qcdpu}{QCD~+~PU\xspace}
\newcommand{\nuoot}{\ensuremath{\nu}-gun~+~OOT~PU\xspace}
\newcommand{\oot}{out-of-time\xspace}
\newcommand{\Oot}{Out-of-time\xspace}
\newcommand{\inti}{in-time\xspace}
\newcommand{\pu}{pileup\xspace}
\newcommand{\Pu}{Pileup\xspace}
\newcommand{\tp}{trigger primitive\xspace}
\newcommand{\Tp}{Trigger primitive\xspace}
\newcommand{\tps}{trigger primitives\xspace}
\newcommand{\Tps}{Trigger primitives\xspace}
\newcommand{\lowet}{$0.5 < E_{\text{T,RH}} \leq 10.0\;\text{GeV}$\xspace}
\newcommand{\loweret}{$0.0 < E_{\text{T,RH}} \leq 10.0\;\text{GeV}$\xspace}
\newcommand{\highet}{$E_{\text{T,RH}} > 10.0\;\text{GeV}$\xspace}
\newcommand{\mintpet}{$E_{\text{T,TP}} > 0.5\;\text{GeV}$\xspace}
\newcommand{\reltpet}{$0.5 < E_{\text{T,TP}} < 128\;\text{GeV}$\xspace}
\newcommand{\ts}{time slice\xspace}
\newcommand{\tss}{time slices\xspace}
\newcommand{\tpet}{\ensuremath{E_{\text{T,TP}}}\xspace}
\newcommand{\rhet}{\ensuremath{E_{\text{T,RH}}}\xspace}
\newcommand{\tprh}{\ensuremath{E_{\text{T,TP}} / E_{\text{T,RH}}}\xspace}
\newcommand\bolden[1]{{\bfseries#1}}
\newcommand\mbolden[1]{{\boldmath\bfseries#1}}
\newcommand\allbold[1]{{\boldmath\textbf{#1}}}

\newcommand{\iphi}{\ensuremath{i\phi}\xspace}
\newcommand{\aiphi}{\ensuremath{|i\phi|}\xspace}
\newcommand{\ieta}{\ensuremath{i\eta}\xspace}
\newcommand{\aieta}{\ensuremath{|i\eta|}\xspace}

\newcommand{\boldcol}[2]{\textcolor{#1}{\bolden{#2}}}
\newcommand{\mboldcol}[2]{\textcolor{#1}{\mbolden{#2}}}
\newcommand\umncell[1]{\cellcolor{UMNMaroon}\mboldcol{UMNGold}{#1}}
\newcommand{\hlink}[2]{\href{#1}{\beamergotobutton{#2}}}
\newcommand{\hclip}[3]{\udensdot{#1}{{\href{#2}{\textcolor{#1}{#3}$\,$\ExternalLink{#1}}}}}

\newcolumntype{C}[1]{>{\centering\arraybackslash}m{#1}}
\newcolumntype{L}[1]{>{\raggedright\arraybackslash}m{#1}}

\newcommand{\redact}[1]{\censor{\phantom{#1}}}

\xspaceaddexceptions{()+} 

\newcommand{\URLsymbol}[1]{$\vcenter{\hbox{\scalebox{0.2}{\begin{tikzpicture}
    \draw[-,line width=3, rounded corners] (0.3,0.3) -- (0.9,0.9) -- (1.1,0.7) -- (0.5,0.1) -- cycle;
    \draw[-,line width=3, rounded corners,#1] (0,0) -- (0.6,0.6) -- (0.8,0.4) -- (0.2,-0.2) -- cycle;
    \draw[-,line width=3, rounded corners] (1,0.8) -- (1.1,0.7) -- (0.5,0.1) -- (0.4,0.2);
\end{tikzpicture}}}}$}

\newcommand{\fade}[1]{%
    \begin{tikzpicture}
        \node[inner sep=0pt,inner ysep=2pt,outer sep=0pt] (A) {#1};
        \filldraw[draw=none,fill=white,fill opacity=0.80] (A.south west) rectangle (A.north east);
    \end{tikzpicture}\par%
}%

\newcommand{\cancel}[1]{%
    \tikz[baseline=(cross.base)]{
        \node[inner sep=1pt,outer sep=0pt] (cross) {#1};
        \draw[solid,line width=0.5pt] ([xshift=0pt,yshift=0.0pt]cross.south west) -- ([xshift=0pt,yshift=0.0pt]cross.north east);
    }%
}%

\newcommand{\crossout}[1]{%
    \tikz[baseline=(cross.base)]{
        \node[inner sep=1pt,outer sep=0pt] (cross) {#1};
        \draw[solid,line width=0.5pt] ([xshift=0pt,yshift=-0.9pt]cross.west) -- ([xshift=0pt,yshift=-0.9pt]cross.east);
    }%
}%

\newcommand{\udensdot}[2]{%
    \tikz[baseline=(todotted.base)]{
        \node[inner sep=1pt,outer sep=0pt] (todotted) {#2};
        \draw[densely dotted,#1] ([xshift=1.5pt,yshift=0.6pt]todotted.south west) -- ([xshift=-1.5pt,yshift=0.6pt]todotted.south east);
    }%
}%

\newcommand{\censor}[1]{%
    \tikz[baseline=(tocensor.base)]{
        \node[inner sep=1pt,outer sep=0pt] (tocensor) {#1};
        \begin{scope}[x={(tocensor.south east)},y={(tocensor.north west)}]
        \filldraw [draw=black,line width=0.05cm,fill=black] (1.0,0.0) rectangle (0.0,1.0);
        \end{scope}
    }%
}%

\newcommand{\ExternalLink}[1]{%
    \tikz[x=1.2ex, y=1.2ex, baseline=-0.05ex]{% 
        \begin{scope}[x=1ex, y=1ex]
            \clip (-0.1,-0.1) 
                --++ (-0, 1.2) 
                --++ (0.6, 0) 
                --++ (0, -0.6) 
                --++ (0.6, 0) 
                --++ (0, -1);
            \path[draw, 
                line width = 0.5, 
                rounded corners=0.5,
                #1] 
                (0,0) rectangle (1,1);
        \end{scope}
        \path[draw, line width = 0.5,#1] (0.5, 0.5) 
            -- (1, 1);
        \path[draw, line width = 0.5,#1] (0.6, 1) 
            -- (1, 1) -- (1, 0.6);
    }%
}%

\newcommand{\backupbegin}{
    \newcounter{finalframe}
    \setcounter{finalframe}{\value{framenumber}}
}
\newcommand{\backupend}{
    \setcounter{framenumber}{\value{finalframe}}
}

\let\footnoterule\relax

%UofMN colors
\definecolor{UMNMaroon}{RGB}{122,0,25}
\definecolor{UMNLightMaroon}{RGB}{153,0,28}
\definecolor{UMNDarkMaroon}{RGB}{92,0,19}

\definecolor{UMNGold}{RGB}{255,204,51}
\definecolor{UMNLightGold}{RGB}{255,215,95}
\definecolor{UMNDarkGold}{RGB}{191,153,38}

\definecolor{UMNStormy}{RGB}{64,77,91}
\definecolor{UMNSunny}{RGB}{0,149,182}
\definecolor{UMNLightGray}{RGB}{213,214,210}

\definecolor{MyRed}{HTML}{E41A1C}
\definecolor{MyBlue}{HTML}{377EB8}
\definecolor{MyGreen}{HTML}{4DAF4A}
\definecolor{MyPurple}{HTML}{984EA3}
\definecolor{MyOrange}{HTML}{FF7F00}

\definecolor{MyPurple2}{HTML}{756BB1}
\definecolor{MyGreen2}{HTML}{31A354}
\definecolor{MyBlue2}{HTML}{3182BD}

\definecolor{Gray1}{HTML}{BDBDBD}
\definecolor{Gray2}{HTML}{969696}
\definecolor{Gray3}{HTML}{525252}
\definecolor{Gray4}{HTML}{252525}
\definecolor{Gray5}{HTML}{000000}

\definecolor{Blue1}{HTML}{9ECAE1}
\definecolor{Blue2}{HTML}{6BAED6}
\definecolor{Blue3}{HTML}{2171B5}
\definecolor{Blue4}{HTML}{08306B}

\definecolor{Green1}{HTML}{A1D99B}
\definecolor{Green2}{HTML}{74C476}
\definecolor{Green3}{HTML}{238B45}
\definecolor{Green4}{HTML}{00441B}

\definecolor{ttcol}{HTML}{6666CD}
\definecolor{qcdcol}{HTML}{05C801}
\definecolor{ttxcol}{HTML}{CD6600}
\definecolor{othcol}{HTML}{990099}
\definecolor{fitcol}{HTML}{58D354}
\definecolor{pullcol}{HTML}{CBCBFE}
\definecolor{sigcol}{HTML}{FF00FF}

\definecolor{mvasig}{HTML}{439643}
\definecolor{mvabkg}{HTML}{98C8E3}

\setbeamercolor{structure}{fg = UMNStormy }

\usetheme{Madrid}

\newcommand\titlefooter{%
\setbeamertemplate{footline}
{%
    \leavevmode%
    \hbox{%
    \begin{beamercolorbox}[wd=1.000000\paperwidth,ht=2.25ex,dp=1ex]{title in head/foot}%
      \centering
      \usebeamerfont{title in head/foot}\insertshorttitle{}
    \end{beamercolorbox}}%
    \vskip0pt%
    \addtocounter{framenumber}{-1}
}%
}

\newcommand\mainfooter{%
\setbeamertemplate{footline}
{%
    \leavevmode%
    \hbox{%
    \begin{beamercolorbox}[wd=.500000\paperwidth,ht=2.25ex,dp=1ex]{author in head/foot}%
      \usebeamerfont{author in head/foot}\hspace*{2ex}\noindent\insertshortauthor\hfill
      \usebeamerfont{title in head/foot}\insertshortsubtitle{}\hspace*{2ex}
    \end{beamercolorbox}%
    \begin{beamercolorbox}[wd=.500000\paperwidth,ht=2.25ex,dp=1ex]{section in head/foot}%
      \usebeamerfont{section in head/foot}\hspace*{2ex}\noindent\insertsection{}\hfill
      \usebeamerfont{date in head/foot}\insertshortdate{}\hspace*{2em}
      \insertframenumber{}\hspace*{2ex}
    \end{beamercolorbox}}%
    \vskip0pt%
}%
}

\setbeamercolor{frametitle}{fg = UMNGold, bg = UMNMaroon }
\setbeamercolor{title}{fg = UMNGold, bg = UMNMaroon }
\setbeamercolor{author in head/foot}{fg = UMNGold, bg = UMNDarkMaroon }
\setbeamercolor{title in head/foot}{fg = UMNGold, bg = UMNMaroon}
\setbeamercolor{date in head/foot}{fg = UMNGold, bg = UMNLightMaroon}
\setbeamercolor{section in head/foot}{fg = UMNGold, bg = UMNLightMaroon}
\setbeamercolor{button}{bg=UMNMaroon,fg=UMNGold}

\setbeamerfont{footnote}{size=\tiny}

\setbeamersize{text margin left=4mm,text margin right=6mm}

\setbeamertemplate{navigation symbols}{} % To remove the navigation symbols from the bottom of all slides
\setbeamertemplate{enumerate items}[default]

\setbeamercolor{itemize item}{fg = UMNMaroon}
\setbeamercolor{itemize subitem}{fg = black}
\setbeamercolor{itemize subsubitem}{fg = UMNMaroon}

\setbeamercolor{enumerate item}{bg = UMNMaroon}
\setbeamercolor{enumerate subitem}{fg = UMNStormy, bg = black}
\setbeamercolor{enumerate subsubitem}{bg = UMNMaroon}

\setbeamertemplate{itemize item}[circle]
\setbeamertemplate{itemize subitem}[triangle]
\setbeamertemplate{itemize subsubitem}{\raisebox{-0.4em}{\scalebox{3.0}{$\cdot$}}}

\setbeamertemplate{itemize/enumerate body begin}{\footnotesize}
\setbeamertemplate{itemize/enumerate subbody begin}{\scriptsize}
\setbeamertemplate{itemize/enumerate subsubbody begin}{\tiny}
